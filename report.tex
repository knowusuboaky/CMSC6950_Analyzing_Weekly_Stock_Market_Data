\documentclass[12pt]{article}
\usepackage[utf8]{inputenc}
\usepackage[english]{babel}
\usepackage{amsmath}
\usepackage{natbib}
\usepackage{adjustbox}
\usepackage{listings}
\usepackage{graphicx}
\usepackage{hyperref}
\usepackage[english]{babel}
\usepackage[T1]{fontenc}
\usepackage[utf8]{inputenc}
\usepackage{caption}
\usepackage{caption,float}
\usepackage[vmargin=3cm,hmargin=3cm]{geometry}

\begin{document}

{\centering

\rule{\textwidth}{1.6pt}\vspace*{-\baselineskip}\vspace*{2pt} 
\rule{\textwidth}{0.4pt}\\[\baselineskip] 
{\LARGE Analyzing the Weekly S\&P Stock Market Data using Logistic Regression}
\rule{\textwidth}{0.4pt}\vspace*{-\baselineskip}\vspace{3.2pt}
\rule{\textwidth}{1.6pt}\\[\baselineskip] 

\vspace{20mm} %5mm vertical space
\scshape % Small caps
CMSC 6950 - Computer Based Research Tools and Applications \\ [\baselineskip]
Term Project \\[\baselineskip] 
6th August, 2020 \\[\baselineskip] 
\vspace{20mm} %5mm vertical space
Submitted by \\[\baselineskip]
{\Large Kwadwo Nyame Owusu-Boakye \\ (201990860) \par}
\vfill
{\itshape Memorial University of Newfoundland \\ St. John's, Canada.\par} 
}

\newpage

{\centering
  \section*{Abstract}
}


\section{Introduction}
The S\&P 500, or basically the S\&P, is a stock exchange index that evaluates the stock performance of 500 huge companies recorded on stock trades in the United States. It is one of the most usually followed equity indicators, and many believe it to be probably the best depictions of the U.S. stock exchange. The normal yearly aggregate return of the index, including profits, since beginning in 1926 has been 9.8\%; be that as it may, there were more than a few years where the index dropped over 30\%. The index has posted yearly increases 70\% of the time (\cite{ref-wiki}) .\\\\
The index is one of the components in calculation of the Conference Board Leading Economic Index, used to predict the course of the economy. The index is related with numerous ticker images, including: \^GSPC, INX, and \$SPX, dependent on marketplace or internet site. The index value is revised each 15 seconds, or 1,559 times per business day, with price upgrade circulated by Reuters (\cite{ref-quote}).



\section{Materials and Methods}


\section{Analysis and Results}







\section{Conclusion}



\begin{thebibliography}{999}

	% Reference 1
	\bibitem[Wikipedia contributors (2020)]{ref-wiki}
	Wikipedia contributors.(2020). S\&P 500 Index. In Wikipedia, The Free Encyclopedia. Online; accessed 28-July-2020, from {\url{https://en.wikipedia.org/w/index.php?title=S\%26P_500_Index&oldid=969280492}}
	
	% Reference 2
	\bibitem[Duggan, Wayne (2019)]{ref-quote}
	Duggan, Wayne (2019). (June 13, 2019). "This Day In Market History: S\&P 500 Quotes Delivered Every 15 Seconds", on June 13, 2019. Benzinga.
	
	% Reference 3
	\bibitem[Wikipedia contributors (2020)]{ref-wiki:xxx}
	Wikipedia contributors.(2020). Logistic regression. In Wikipedia, The Free Encyclopedia Online; accessed 28-July-2020, from  {\url{https://en.wikipedia.org/w/index.php?title=Logistic\_regression\&oldid=967311267}}
	
	% Reference 4
	\bibitem[Vallat, R. (2018)]{ref-software}
	Vallat, R. (2018). Pingouin: statistics in Python. Journal of Open Source Software, 3(31), 1026, {\url{https://doi.org/10.21105/joss.01026}}
	
	% Reference 5
	\bibitem[Haghighi et al., (2018)]{ref-softwaree}
	Haghighi et al., (2018). PyCM: Multiclass confusion matrix library in Python. Journal of Open Source Software, 3(25), 729.
	{\url{https://doi.org/10.21105/joss.00729}}
\end{thebibliography}
\end{document}

